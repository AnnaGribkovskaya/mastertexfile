\chapter{Coupled Cluster method} \label{ch:coupled_cluster}
Coupled cluster (CC) method for quantum chemistry was first introduced
by J. Cizek in the late 1960s \cite{cizekCorrelationProblemsAtomic}. A
few years later he published a new article on the topic in collaboration
with and J. Paldus \cite{cizekCorrelationProblemAtomic1966}. CC is an
\textit{ab initio} numerical method widely used for approximate
solution of the electronic Schr\"{o}dinger equation because it is both
reliable and computationally affordable. This section is based on a
very detailed and comprehensive overview of the method provided by
Crawford and Schaefer in
\cite{crawfordIntroductionCoupledCluster2007}. The coupled cluster method
is based on the same basic concepts that are underlying many other
many-body methods, such as many-body perturbation theory and full
configuration interaction. The main critical difference is the use of the
"exponential ansatz" of the wave function which is discussed
below.\\ In this chapter we discuss some critical ideas for the CC method,
such as cluster expansion of the wave function , exponential ansatz,
Campbell-Baker-Hausdorff (BCH) expansion, second quantization and
particle-hole formalism, normal-ordering and correlation operator in
application for the CC method. Some of these ideas we have already mentioned
in our previous chapters and some are completely new. The CC method can be
described in two different ways using algebraic and diagrammatic
formalisms. Both are correct and provide the same results, but
the diagrammatic one is way faster to apply. However, for the understanding of the
method and its origin we need to begin with algebraic form of the
equations.\\
\section{Cluster functions and Exponential Anzats} 
As we have already discussed above, the Slater Determinant can be used to describe a wave function of the electrons. In Dirac notations it can be written as follows:
\begin{equation}
\Phi_0 =  \ket{\phi_i(\bm{x_1})\phi_j(\bm{x_2}) ...\phi_l(\bm{x_n})}
\end{equation}
here $\phi_i(\bm{x_1})$ is a one-electron wave function that describes the motion of each electron separately, the $\bm{x_1}$ is a vector of coordinates, both spatial and spin. \\
Keeping in mind that we are working with fermions, the electronic wave function should be antisymmetric. \\
Such a description has some drawbacks, for example it fails to take into account the instantaneous interactions that keep
apart electrons with opposite spins. For more details see for example   \cite{bartlettApplicationsPostHartree2007} and \cite{ModernQuantumChemistry}. \\
The most important idea of the CC method is the exponential expansion of the wave function,
\begin{equation}\label{eq:exponential ansatz}
\ket{\Psi}=e^{\hat{T}}\ket{\Phi_0}.
\end{equation}
Here we have defined the cluster operator $\hat{T}$ is cluster operator:

\begin{align} \label{eq:T}
\hat{T}   &= \hat{T_1}+ \hat{T_2}+ \hat{T_3}...\hat{T_n},  \\
\hat{T_1} &= \sum_{i,a} t_i^a c_a^\dagger c_i, \label{eq:T1}\\
\hat{T_2} &= \frac{1}{4} \sum_{i,j,a,b} t_{i,j}^{a,b} c_a^\dagger a_b^\dagger c_j c_i,\\
\hat{T_n} &= \bigg(\frac{1}{n!}\bigg)^2 \sum_{i,j \dots a,b\dots } t_{i,j\dots}^{a,b\dots}  c_a^\dagger c_b^\dagger \dots c_j c_i.\\
\end{align}
Here the values $t_i^a $ and $t_{i,j}^{a,b}$ are called cluster amplitudes. \\
After introducing the exponential ansatz we need a recipe to determine the cluster amplitudes. In order to do this we start with electronic Schr\"{o}dinger equation ($\ref{eq:Ham in 2q}$) and use equation ($\ref{eq:exponential ansatz}$) to approximate the exact wave function:
\begin{equation}
	\hat{H}e^{\hat{T}}\ket{\Phi_0}=Ee^{\hat{T}}\ket{\Phi_0}.
\end{equation}
Using the \textit{intermediate normalization} $\braket{\Phi_0|\Psi_{CC}}=1$ and  the equation above we can immediately get the energy:
\begin{equation}\label{eq:energy simple cc}
\bra{\Phi_0}\hat{H}e^{\hat{T}}\ket{\Phi_0}=E.
\end{equation}
Now we want to use the fact that exponentiated operator can be expressed in terms of a power series:
 \begin{equation}
 e^{\hat{T}}= 1+ \hat{T}+\frac{1}{2!}\hat{T}^2+ ...
 \end{equation}
If we now insert this equation into equation ($\ref{eq:energy simple cc}$) and simplify a bit we obtain:
\begin{equation}
 \bra{\Phi_0}\hat{H}\ket{\Phi_0} +  \bra{\Phi_0}\hat{H}\hat{T}\ket{\Phi_0} +\bra{\Phi_0}\hat{H}\frac{1}{2!}\hat{T}^2\ket{\Phi_0}  = E.
\end{equation}
In the energy equation above the exponential expansion is truncated after $\hat{T}^2$. This fact is often referred to as \textit{natural truncation} of the coupled cluster energy equation. It's important here that the truncation occurs due to the fact that the Hamiltonian in our case is at most two-body operator and the cluster operator is at least of a one-body character. It this case the matrix elements of the Hamiltonian are zero for all determinants that differ more then two orbitals. This truncation does not depend on the number of particles or any other parameter, it only occurs due to the form of Hamiltonian. \\
The amplitude expressions can be computed using the same projection technique, only instead of projecting into the reference state we will be projecting into an excited determinant given as 
\begin{equation}\label{eq:ampl simple}
\bra{\Phi_{ij...}^{ab...}}\hat{H} e^{\hat{T}}\ket{\Phi_0} = E\bra{\Phi_{ij...}^{ab...}} e^{\hat{T}}\ket{\Phi_0}.
\end{equation}

\subsection{Baker-Campbell-Hausdorff Expansion} \label{sec:Hausdorf}
The expressions for energy and amplitudes in the section above are not useful for practical computations. In order to obtain equations that can be efficiently implemented in computer programs we are going to use so called similarly transformed Hamiltonian. It is a well known procedure in quantum mechanics. 

For the energy equation we get:
\begin{equation}\label{eq:energy_cc1}
\bra{\Phi_0} e^{\hat{-T}} \hat{H} e^{\hat{T}} \ket{\Phi_0}   = E. 
\end{equation}
For amplitudes equation we get:
\begin{equation}\label{eq:amplitudesgeneral}
\bra{\Phi_{ij...}^{ab...}}e^{\hat{-T}} \hat{H} e^{\hat{T}} \ket{\Phi_0}   = 0.
\end{equation} 
These equations are equivalent to ($\ref{eq:ampl simple}$) and ($\ref{eq:energy simple cc}$) derived in the sections above. However there are two significant advantages this transformation provides to the method. Equations for the amplitudes are now not coupled to the energy equation and we may use the so-called Campbell-Baker-Hausdorff formula and rewrite the expression for $ e^{\hat{-T}} \hat{H} e^{\hat{T}} $ in terms of linear combinations of nested commutators of $ \hat{H} $ and $\hat{T}$ as
\begin{equation}
e^{\hat{-T}} \hat{H} e^{\hat{T}} = \hat{H} + [\hat{H},\hat{T}] + \frac{1}{2!}[[\hat{H},\hat{T}],\hat{T}] + \frac{1}{3!}[[[\hat{H},\hat{T}],\hat{T}],\hat{T}] + ...
\end{equation} 
Here we need to make a very important remark regarding the transformation we have done. As it has been mentioned in chapter one, quantum mechanics demands that observables are expectation values of hermitian operators (Postulate $\ref{postulat2}$). The operator  $e^{\hat{-T}} \hat{H} e^{\hat{T}} $ is obviously not hermitian, 
\begin{equation}
(e^{\hat{-T}} \hat{H} e^{\hat{T}})^\dagger \neq e^{\hat{-T}} \hat{H} e^{\hat{T}}
\end{equation} 
However, the eigenvalue spectrum of this operator is identical to the original Hamiltonian operator. For details please refer to \cite{kutzelniggAlmostVariationalCoupled}.

\section{Coupled Cluster Equations}
\subsection{Energy Equation}
In the previous section $\ref{sec:Hausdorf}$ we have introduced
Hausdorff expansion for similarly transformed Hamiltonian. Now we will
use it to derive the energy equation. In case of CCSD (meaning
$\hat{T}=\hat{T}_1+\hat{T}_2$) the similarly transformed normal
ordered Hamiltonian $\bar{H}=e^{\hat{-T}} \hat{H}_\text{N}
e^{\hat{T}}$ can be written as:

\begin{eqnarray}\label{eq:hausdorf_ccsd}
 \hat{H}_\text{N} + [\hat{H}_\text{N},\hat{T}_1+\hat{T}_2] + \frac{1}{2!}[[\hat{H}_\text{N},\hat{T}_1+\hat{T}_2],\hat{T}_1+\hat{T}_2] + ...= \nonumber\\
 \hat{H}_\text{N} + [\hat{H}_\text{N},\hat{T}_1] + [\hat{H}_\text{N},\hat{T}_2] + \frac{1}{2}[[\hat{H}_\text{N},\hat{T}_1],\hat{T}_1] +\nonumber\\ \frac{1}{2}[[\hat{H}_\text{N},\hat{T}_2],\hat{T}_2] +  \frac{1}{2}[[\hat{H}_\text{N},\hat{T}_1],\hat{T}_2] + \frac{1}{2}[[\hat{H}_\text{N},\hat{T}_2],\hat{T}_1] + ...
\end{eqnarray}

We now want to insert ($\ref{eq:hausdorf_ccsd}$) into ($\ref{eq:energy_cc1}$). The $\hat{H}_\text{N}$ term is trivial as $\braket{\Phi_0|\hat{H}_\text{N}|\Phi_0}=0$.\\
The next term is a bit more interesting:
\begin{eqnarray}\label{eq:firstterm}
 [\hat{H}_\text{N},\hat{T}_1] =  [\hat{F}_\text{N},\hat{T}_1] +  [\hat{V}_\text{N},\hat{T}_1].
\end{eqnarray}
Using ($\ref{eq:T1}$) and the formulation of the Fock operator from ($\ref{eq:Fock_operator}$) we need to deal with products of operator strings and we may immediately re-write them using Wick's theorem:
\begin{eqnarray}
\{c_p^\dagger c_q \}\{c_a^\dagger c_i \} =\{c_p^\dagger c_q c_a^\dagger c_i \} + \wick{ \{\c1 c_p^\dagger c_q c_a^\dagger \c1 c_i \} } + \wick{ \{ c_p^\dagger \c1 c_q \c1 c_a^\dagger  c_i \} } +
 \wick{ \{ \c2 c_p^\dagger \c1 c_q \c1 c_a^\dagger \c2  c_i \} } = \nonumber\\
 \{c_p^\dagger c_q c_a^\dagger c_i \} + \delta_{pi} \{ c_q c_a^\dagger\} +\delta_{qa} \{  c_p^\dagger c_i \} +  \delta_{pi}\delta_{qa}\\
 \{c_a^\dagger c_i \}\{c_p^\dagger c_q \}= \{c_p^\dagger c_q c_a^\dagger c_i \}.
\end{eqnarray}
Using this the next term ($\ref{eq:firstterm}$), the expansion ($\ref{eq:hausdorf_ccsd}$) may be written as
\begin{eqnarray}
\bra{\Phi_0}[\hat{F}_\text{N},\hat{T}_1] \ket{\Phi_0}=\sum_{ia} f_{ia}t_i^a,\\
\bra{\Phi_0}[\hat{V}_\text{N},\hat{T}_1] \ket{\Phi_0}=0.
\end{eqnarray}
The next term is:
\begin{eqnarray}\label{eq:secondterm}
[\hat{H}_\text{N},\hat{T}_2]=[\hat{F}_\text{N},\hat{T}_2] +  [\hat{V}_\text{N},\hat{T}_2].
\end{eqnarray} 
Following the same procedure as before we get:
\begin{eqnarray}
\bra{\Phi_0}[\hat{F}_\text{N},\hat{T}_2] \ket{\Phi_0}=0\\
\bra{\Phi_0}[\hat{V}_\text{N},\hat{T}_2] \ket{\Phi_0}=\frac{1}{4}\sum_{ijab}\braket{ij|\hat{v}|ab}t_{ij}^{ab}
\end{eqnarray}
And for the next term we have:
\begin{eqnarray}
\frac{1}{2}[[\hat{H}_\text{N},\hat{T}_1],\hat{T}_1]= \frac{1}{2}\hat{H}_\text{N}\hat{T}_1^2 -\hat{T}_1\hat{H}_\text{N}\hat{T}_1 + \frac{1}{2}\hat{T}_1^2\hat{H}_\text{N}=\nonumber \\
 \frac{1}{2}\hat{F}_\text{N}\hat{T}_1^2 +  \frac{1}{2}\hat{V}_\text{N}\hat{T}_1^2 
 -\hat{T}_1\hat{F}_\text{N}\hat{T}_1  -\hat{T}_1\hat{V}_\text{N}\hat{T}_1 + 
 \frac{1}{2}\hat{T}_1^2\hat{F}_\text{N} +  \frac{1}{2}\hat{T}_1^2\hat{V}_\text{N}
\end{eqnarray}
Let us consider this one more carefully, because it can help us to state a so-called connected contraction theorem, resulting in 
\begin{eqnarray}
\hat{F}_\text{N}\hat{T}_1^2 = \sum_{ijab} \sum_{pq}f_{pq}t_i^at_j^b(\{c_p^\dagger c_q c_a^\dagger c_i c_b^\dagger c_j \}+ \wick{\{\c1 c_p^\dagger c_q c_a^\dagger \c1 c_i c_b^\dagger c_j \}}  +\wick{\{\c1 c_p^\dagger c_q c_a^\dagger  c_i c_b^\dagger \c1c_j \}} + \nonumber \\
\wick{\{ c_p^\dagger \c1 c_q \c1c_a^\dagger  c_i c_b^\dagger c_j \}} + \wick{\{ c_p^\dagger \c1 c_q c_a^\dagger  c_i \c1 c_b^\dagger c_j \}} + \wick{\{ \c2 c_p^\dagger  \c1 c_q \c1 c_a^\dagger \c2 c_i  c_b^\dagger  c_j \}} + \nonumber \\ \wick{\{ \c2 c_p^\dagger  \c1 c_q \c1 c_a^\dagger  c_i  c_b^\dagger \c2 c_j \}}+ \wick{\{ \c2 c_p^\dagger  \c1 c_q  c_a^\dagger \c2 c_i \c1 c_b^\dagger  c_j \}}+  \wick{\{ \c2 c_p^\dagger  \c1 c_q  c_a^\dagger  c_i \c1 c_b^\dagger \c2 c_j \}})
\end{eqnarray}
As we can see there is no fully contracted terms in the expression above. Two other terms can be written as:
\begin{align*}
\hat{T}_1\hat{F}_\text{N}\hat{T}_1 &= \nonumber \\ 
\sum_{ijab} t_i^at_j^b \bigg( &\sum_{pq}f_{pq} \{c_a^\dagger c_ic_p^\dagger c_q  c_b^\dagger c_j \} +  \sum_{q}f_{jq}\{c_a^\dagger c_i  c_q  c_b^\dagger \} +  \sum_{p}f_{pb}\{c_a^\dagger c_i   c_p^\dagger c_j \} + f_{jb}\{c_a^\dagger c_i \}\bigg)\nonumber \\
\hat{T}_1^2\hat{F}_\text{N} =&\sum_{ijab}  \sum_{pq}f_{pq}  t_i^at_j^b \{c_a^\dagger c_ic_b^\dagger c_j  c_p^\dagger c_q   \} \nonumber \\
\end{align*}
Non of the above terms contribute to the energy expectation
value. However the term $[\hat{F}_\text{N},\hat{T}_1] $ does. Using
this, we may now formulate the connected contraction theorem:
\begin{theorem} Connected Cluster Theorem \\ \label{theo:connected}
The contribution to the energy expectation value from the terms in the Baker-Campbell-Hausdorff expansion of similarity transformed normal ordered Hamiltonian is non zero only if the the $\hat{H}_\text{N}$ has at least one contraction with every cluster operator on the right side.\\
Such connected contractions are denoted with subscript $(...)_c$.
\end{theorem}
Using this theorem one can say that as soon as the Hamiltonian has at most four operators the Hausdorff expansion can have at most four cluster operators. This eventually makes the derivation of the energy and amplitude equations much easier.\\
Let's continue with the  derivation of the energy equation using the connected contractions. The contribution from $(\hat{F}_\text{N}\hat{T_2})_c$ is zero by construction. Due to the absence of fully contracted terms ($\hat{F}_\text{N}$ contains two operators and $\hat{T_2}$ has four). The contribution from $(\hat{V}_\text{N}\hat{T_2})_c$ is non zero, because we have two normal ordered operator strings of the same size here.:
\begin{equation}
\bra{\Phi_0}(\hat{V}_\text{N}\hat{T}_2)_c \ket{\Phi_0}=\frac{1}{4}\sum_{ijab} \braket{ij||ab}t_{ij}^{ab}.
\end{equation}
The next non-zero contribution comes from term $(\hat{V}_\text{N}\hat{T_1}^2)_c$:
\begin{equation}
\bra{\Phi_0}(\hat{V}_\text{N}\hat{T}_1^2)_c \ket{\Phi_0}=\sum_{ijab} \braket{ij||ab}t_{i}^{a}t_{j}^{b}.
\end{equation}
Gathering all the terms carefully one obtains the following expression for the correlation energy:
\begin{equation}
E_{corr}= E_{CCSD} - E_{HF}= \sum_{ia} f_{ia}t_i^a + \frac{1}{4}\sum_{ijab} \braket{ij||ab}t_{ij}^{ab} +\frac{1}{2}\sum_{ijab} \braket{ij||ab}t_{i}^{a}t_{j}^{b}.
\end{equation}
Or in case of coupled clusters doubles simply:
\begin{equation}
E_{corr}= E_{CCD} - E_{HF}= \frac{1}{4}\sum_{ijab} \braket{ij||ab}t_{ij}^{ab}.
\end{equation}
\subsection{Amplitudes equations}
Above we presented a general form for amplitudes equation in
($\ref{eq:amplitudesgeneral}$). Depending on the truncation of the cluster
operator we may have different types of coupled cluster methods:
coupled cluster singles and doubles (CCSD), coupled cluster doubles
(CCD) or coupled cluster singes, doubles and triples (CCSDT). The
general form for amplitude equations ($\ref{eq:amplitudesgeneral}$)
then can be used to find singly-excited amplitudes $t_{i}^{a}$ and
doubly-excited amplitudes $t_{ij}^{ab}$, that is


\begin{eqnarray}
\bra{\Phi_{ij}^{ab}}e^{\hat{-T}} \hat{H} e^{\hat{T}} \ket{\Phi_0}   = 0 \xRightarrow{~\textsf{CCD}~} \text{ for  $t_{ij}^{ab}$},
\end{eqnarray}

\begin{eqnarray}
\begin{aligned}
\bra{\Phi_{i}^{a}}e^{\hat{-T}} \hat{H} e^{\hat{T}} \ket{\Phi_0}   = 0  \\
\bra{\Phi_{ij}^{ab}}e^{\hat{-T}} \hat{H} e^{\hat{T}} \ket{\Phi_0}   = 0  \end{aligned}
\xRightarrow{~\textsf{CCSD}~}
\begin{aligned}
\text{ for  $t_{i}^{a}$} \\
\text{ for  $t_{ij}^{ab}$} \end{aligned}
\end{eqnarray}

\begin{eqnarray}
\begin{aligned}
\bra{\Phi_{i}^{a}}e^{\hat{-T}} \hat{H} e^{\hat{T}} \ket{\Phi_0}   = 0  \\
\bra{\Phi_{ij}^{ab}}e^{\hat{-T}} \hat{H} e^{\hat{T}} \ket{\Phi_0}   = 0 \\
\bra{\Phi_{ijk}^{abc}}e^{\hat{-T}} \hat{H} e^{\hat{T}} \ket{\Phi_0}   = 0  \\ \end{aligned}
\xRightarrow{~\textsf{CCSDT}~}
\begin{aligned}
\text{ for  $t_{i}^{a}$} \\
\text{ for  $t_{ij}^{ab}$} \\
\text{ for  $t_{ijk}^{abc}$} \end{aligned}
\end{eqnarray}
And so on.\\
The equations for the amplitudes are derived using the same approach as for the energy, we just write all operators as normal ordered strings and look for fully contracted and connected clusters. Additionally we need to write an excited determinant as an operator string also:
\begin{eqnarray}
\bra{\Phi_{ij}^{ab}} = \bra{\Phi_0} \{  c_i^\dagger c_j^\dagger  c_b  c_a \}.
\end{eqnarray}
The derivation is rather simple, though quite time-consuming due to the number of terms in Baker-Campbell-Hausdorff expansion to take care of. We are not going to tackle all of them and will just present some as examples.\\
We are going to look into the CCD approximation here, thus we will only consider terms that contribute to the amplitudes for $\hat{T}_2$.\\
The one-body term does not contribute because it is not possible to construct fully contracted terms, so the only term that is able to produce such contractions is:
\begin{eqnarray}
\braket{\Phi_{ij}^{ab}|(\hat{F}_\text{N}+\hat{V}_\text{N})|\Phi_0} \rightarrow \braket{\Phi_{ij}^{ab}|(\hat{V}_\text{N})|\Phi_0},\nonumber \\
\braket{\Phi_{ij}^{ab}|(\hat{V}_\text{N})|\Phi_0} = \frac{1}{4}\sum_{pqrs} \braket{pq||rs}\braket{\Phi_0| \{  c_i^\dagger c_j^\dagger  c_b  c_a \} \{  c_p^\dagger c_q^\dagger  c_s  c_r \} |\Phi_0}=\nonumber \\
 \frac{1}{4}\sum_{pqrs} \braket{pq||rs} \bigg( \wick{ \{ \c4 c_i^\dagger \c3 c_j^\dagger \c2 c_b \c1 c_a \c1  c_p^\dagger \c2 c_q^\dagger \c3 c_s \c4 c_r \}  } 
 + \wick{ \{ \c4 c_i^\dagger \c3 c_j^\dagger \c2 c_b \c1 c_a \c2  c_p^\dagger \c1 c_q^\dagger \c3 c_s \c4 c_r \}  } + \nonumber \\
  \wick{ \{ \c4 c_i^\dagger \c3 c_j^\dagger \c2 c_b \c1 c_a \c1  c_p^\dagger \c2 c_q^\dagger \c4 c_s \c3 c_r \}  }+\wick{ \{ \c4 c_i^\dagger \c3 c_j^\dagger \c2 c_b \c1 c_a \c2  c_p^\dagger \c1 c_q^\dagger \c4 c_s \c3 c_r \}  }  \bigg) =\braket{ab||ij}.
\end{eqnarray}
The contribution from other terms are obtained in a similar manner, one just needs to keep in mind that for terms with tehe cluster operator, like $\braket{\Phi_{ij}^{ab}|(\hat{F}_\text{N}\hat{T}_2)_c|\Phi_0}$ and $\braket{\Phi_{ij}^{ab}|(\hat{V}_\text{N}\hat{T}_2)_c|\Phi_0}$ the Connected Cluster Theorem $\ref{theo:connected}$ must be satisfied.\\
We are not providing the derivation any further, because it has been done already in many other articles, for example in an article by Crawford and Schaefer  \cite{crawfordIntroductionCoupledCluster2007}, and several master theses, for example in M. Lohne's thesis \cite{lohneCOUPLEDCLUSTERSTUDIESQUANTUMa}.
The final expression for the amplitudes in case of the CCD approximation is the following:
\begin{eqnarray}\label{eq:CCD_ampl}
0 =\braket{ab||ij} + \sum_{c}(f_{bc}t_{ij}^{ac}-f_{ac}t_{ij}^{bc} ) - \sum_{k}(f_{kj}t_{ik}^{ab}-f_{ki}t_{jk}^{ab} ) +\frac{1}{2} \sum_{cd}\braket{ab||cd}t_{ij}^{cd} 
\nonumber &+\\
 \frac{1}{2} \sum_{kl}\braket{kl||ij}t_{kl}^{ab}  + P(ij)(ab)  \sum_{kc}\braket{kb||cj}t_{ik}^{ac} + 
  P(ij) \sum_{klcd}\braket{kl||cd}t_{ik}^{ac} t_{lj}^{db} \nonumber & + \\
 \frac{1}{4} \sum_{klcd}\braket{kl||cd}t_{ij}^{cd} t_{kl}^{ab}  
- \frac{1}{2} P(ij) \sum_{klcd}\braket{kl||cd}t_{ik}^{dc}t_{lj}^{ab}- \frac{1}{2} P(ab) \sum_{klcd}\braket{kl||cd}t_{lk}^{ac}t_{ij}^{db}.&  
\end{eqnarray}

The only thing left now is to re-write the equation ($\ref{eq:CCD_ampl}$) in an iterative way. This can be done as follows:
\begin{align} \label{eq:ccditer}
   \sum_{c}(f_{bc}t_{ij}^{ac}-f_{ac}t_{ij}^{bc} ) - \sum_{k}(f_{kj}t_{ik}^{ab}-f_{ki}t_{jk}^{ab} )\rightarrow \nonumber \\
   \rightarrow f_{bb}t_{ij}^{ab} - f_{aa}t_{ij}^{ba}
  	- f_{jj}t_{ij}^{ab} + f_{ii}t_{ji}^{ab} = - \epsilon_{ij}^{ab} t_{ij}^{ab},
\end{align} 	
 here $ \epsilon_{ij}^{ab} = \epsilon_i + \epsilon_j - \epsilon_a
 - \epsilon_b $.
   \begin{align}
& \epsilon_{ij}^{ab} t_{ij}^{ab(n)} = \braket{ab||ij}  +\frac{1}{2} \sum_{cd}\braket{ab||cd}t_{ij}^{cd(n-1)}
   +	\frac{1}{2} \sum_{kl}\braket{kl||ij}t_{kl}^{ab(n-1)} 
   + P(ij)(ab)  \sum_{kc}\braket{kb||cj}t_{ik}^{ac(n-1)}  \nonumber \\
&  +	P(ij) \sum_{klcd}\braket{kl||cd}t_{ik}^{ac}(n-1) t_{lj}^{db(n-1)}   + 
  	\frac{1}{4} \sum_{klcd}\braket{kl||cd}t_{ij}^{cd(n-1)} t_{kl}^{ab(n-1)}  \\
&  	- \frac{1}{2} P(ij) \sum_{klcd}\braket{kl||cd}t_{ik}^{dc(n-1)}t_{lj}^{ab(n-1)}- \frac{1}{2} P(ab) \sum_{klcd}\braket{kl||cd}t_{lk}^{ac(n-1)}t_{ij}^{db(n-1)}, \nonumber
\end{align}
here $n$ is number of iteration. 

