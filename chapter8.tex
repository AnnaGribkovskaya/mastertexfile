\chapter{Results and discussion} \label{sec:Results}

\section{The CCD results}
As we have already mentioned the HEG is a very convenient system to study various many-body methods. In particular we have found many other researches who have implemented the CCD approximation for such system. That allow us to test our solution against results obtained in many other works on the topic. In particular we have compared our results with some other students from our department, namely with Miller \cite{millerAntumMechanicalStudies}, Hansen \cite{hansenCoupledClusterStudies} and \cite{gustavbaardsenCoupledclusterTheoryInfinite2014}. Table $\ref{tab:CCDcompar}$ present the comparison of the results. As one can see we manage to reproduce their results up to a chosen \textbf{tolerance} for CCD (in this case $10^{-6}$). 


\begin{landscape}
	\begin{table}[h]
		\centering

		\begin{tabular}{rrllll}
			$r_s$ & States & $\Delta E_{CCD}^{Miller}$ &$\Delta E_{CCD}^{Hansen}$  &$\Delta E_{CCD}^{Baardsen}$ & $\Delta E_{CCD}$\\
			\hline
			\hline
			1.0 & 54 & -0.3178228436889338 & -0.317822843688933   &           & -0.3178230699319593  \\
			1.0 & 66 & -0.3926965898061968 & -0.3926965898061966  &           & -0.3926968074770886  \\
			1.0 & 114 & -0.4479105961757175 & -0.4479105961757175 & -0.447909 & -0.4479109389185165  \\
			1.0 & 162 & -0.4805572589306421 & -0.4805572589306416 &           & -0.4805570782443642  \\
			1.0 & 186 & -0.4855229317521320 & -0.4855229317521318 & -0.485523 & -0.4855227418241649  \\
			1.0 & 246 & -0.4929245740023971 & -0.4929245740023975 &           & -0.4929243692209991  \\
			1.0 & 294 & -0.4984909094066806 & -0.4984909094066818 &           & -0.4984906939593084  \\
			1.0 & 342 & -0.5019526761547777 & -0.5019526761547779 &           & -0.5019524529049425  \\
			1.0 & 358 & -0.5025196736076414 & -0.502519673607641  & -0.502523 & -0.5025194488388953  \\ \hline
			0.5 & 114 & -0.5120153541478306 & -0.5120153541478306 & -0.512015 & -0.5120152296730573  \\
			0.5 & 186 & -0.5114620616957333 &                     & -0.553329 & -0.553329520936615   \\
			0.5 & 342 & -0.5729645498903680 & -0.572964549890367  &           & -0.572964399507112   \\
			0.5 & 358 & -0.5297417436322546 &                     & -0.573678 & -0.5736804143578936  \\ \hline    
			2.0 & 114 & -0.3577968843144996 & -0.3577968843144996 & -0.357798 & -0.3577955282575226  \\
			2.0 & 342 & -0.4014136184665558 & -0.4014136184665555 &           & -0.4014117905655014  \\
		\end{tabular}
				\captionsetup{width=1.1\textwidth}
				\caption{Our results for correlation energy in Hartree units for 14 electrons obtained by CCD solver $\Delta E_{CCD}$, presented with results from Miller \cite{millerAntumMechanicalStudies}, results from Hansen \cite{hansenCoupledClusterStudies} and results from Baardsen \cite{gustavbaardsenCoupledclusterTheoryInfinite2014}. 
				} \label{tab:CCDcompar}
	\end{table}
\end{landscape}
Another test is to compare our results with those from solver presented in \cite{hjorth-jensenAdvancedCourseComputational2017}. Table $\ref{tab:unittest}$ present data for a small number of electrons ($N_e=2$) and relatively small number of states ($N_s$). As one can see from the table the results for the energy are almost the same (up to machine precision).
\begin{table}[h]
	\centering

	\begin{tabular}{rlll}
		n & $\Delta E_{CCD}$     &   $\Delta E_{CCD}^{test}$  \\ \hline
		0 & -0.01833394188507821 & -0.0183339418850782  \\
		1 & -0.01864313296809337 & -0.01864313296809333 \\
		2 & -0.01860449158525421 & -0.01860449158525414 \\
		3 & -0.01860932269795203 & -0.01860932269795206 \\
		4 & -0.01860871870720909 & -0.01860871870720907 \\
	\end{tabular}
		\captionsetup{width=1\textwidth}
		\caption{Our results for correlation energy $\Delta E_{CCD}$  for $N_e=2$, $r_s=1$ and $N_s=66$,   tested against results  $\Delta E_{CCD}^{test}$ from \cite{hjorth-jensenAdvancedCourseComputational2017} CCD solver for the HEG. Mixing parameter is equal one. All values are in Hartree units.}
	\label{tab:unittest}
\end{table}
\subsubsection{Complete basis set and Thermodynamical limit}
In thermodynamical limit the number of particles should approaches infinity. However this is not the case for a real-life computations. And even for the finite number of particles it is hard to perform computations for large numbers, because even best many-body methods usually rapidly scale with size. A solution here is to perform computations for some different number of particles and then use extrapolation technique in order to obtain the limit.\\
In complete basis set limit (CBS) the number of single-particle wave functions should approach infinity. However this is also cannot be achieved for real-life computations. Normally we have to limit our basis to a certain number of basis functions. On Fig. $\ref{fig:CBS}$ one can see how the extrapolation technique is applied to obtain get the correlation energy in the CBS limit for HEG. Here we have used a so-called single-point extrapolation presented by Shepherd et al. in \cite{shepherdInvestigationFullConfiguration2012}. We have obtained value $-0.5142$ for the correlation energy, while Shepherd got $-0.5325(4)$. The result for CCQMC obtained by Thom is $-0.5155(3)$ \cite{spencerDevelopmentsStochasticCoupled2016}.

\begin{figure}[ht!]
	\centering
	\includegraphics[width=0.8\linewidth]{CCDvsCCDT}
	\caption{The correlation energy plotted against different $r_s$ the  electron gas, for $N_e=14$ and 54 single-particle states. CCDT results were taken from \cite{hansenCoupledClusterStudies}}
	\label{fig:CCDvsCCDT}
\end{figure}

\begin{figure}[ht!]
	\centering
	\includegraphics[width=0.8\linewidth]{cbs}
	\caption{The CBS limit for the 3D electron gas obtained by extrapolation of a second degree polynomial in $N_{states}^{-1}$. Here Miller's \cite{millerAntumMechanicalStudies} results for higher number of states([406 - 2090] were used. $N_e=14$ and limit $N_{states} \rightarrow \infty$ is -0.514204  }
	\label{fig:CBS}
\end{figure}



\subsubsection{Results for quantum dot}
We do not obtain a lot of results for QD as soon as our main system was HEG. However CCD solver we have developed for the HEG can be also used to compute ground state properties for QD. The results for QD is present in tables $\ref{tab:resultsHF}$ and  $\ref{tab:resultsCCD}$ and have been tested against the results obtained by Lohne in his master thesis for CCSD \cite{lohneCOUPLEDCLUSTERSTUDIESQUANTUM}. For the Hartree-Fock we obtain exactly the same results within the given precision ($10^{-6}$). 
The exact energy in case of  2 electrons and oscillator potential $\omega=1$ has been obtained analytically by Taut \cite{tautTwoElectronsExternal1993a}. It is equal exactly 3 a.u. We do not obtain this value, however the correlation energy for CCD brings us closer to this result. Unfortunately our program for QD is rather slow and we just tested it for the very simple case of two electrons. In the next Chapter we will discuss how it can be improved in order to get a significant speed up.  

\begin{table}[h!]
	\begin{center}
		\begin{tabular}{|c c| c c| c c| c c|}
			%\cline{1-12}
			%         & \multicolumn{1}{c}{N=2} & \multicolumn{5}{c}{N=6}  \\
			%\cline{1-12}
			\hline
			\multirow{2}{*}{} & 
			\multicolumn{1}{c}{$\omega$=0.1} \vline& 
			\multicolumn{2}{c}{$\omega$=0.5} \vline&
			\multicolumn{2}{c}{$\omega$=1} \vline&
			\multicolumn{2}{c}{$\omega$=2} \vline\\
			\hline
			$R$  & $\Delta E_{corr}$ & $R$ & $\Delta E_{corr}$ & $R$  & $\Delta E_{corr}$ &$R$ &  $\Delta E_{corr}$  \\
			\hline
			$  3 $   & $-0.084668$  &$ 3 $  & $-0.117877$  &$  3 $   & $-0.123643$   &$ 3$  & $-0.125520$   \\
			$  4 $   & $-0.084892$  &$ 4 $  & $-0.125975$  &$  4 $   & $-0.137418$   &$ 4$  & $-0.144715$   \\
			$  5 $   & $-0.082373$  &$ 5 $  & $-0.129695$  &$  5 $   & $-0.143977$   &$ 5$  & $-0.153792$   \\
			$  6 $   & $-0.082521$  &$ 6 $  & $-0.131944$  &$  6 $   & $-0.147998$   &$ 6$  & $-0.159730$   \\
			$  7 $   & $-0.082579$  &$ 7 $  & $-0.133271$  &$  7 $   & $-0.150504$   &$ 7$  & $-0.163497$   \\
			$  8 $   & $-0.082654$  &$ 8 $  & $-0.134251$  &$  8 $   & $-0.152288$   &$ 8$  & $-0.166194$   \\
			$  9 $   & $-0.082708$  &$ 9 $  & $-0.134938$  &$  9 $   & $-0.153566$   &$ 9$  & $-0.168154$   \\
			$  10$   & $-0.082749$  &$ 10$  & $-0.135473$  &$  10$   & $-0.154552$   &$ 10$ & $-0.169664$   \\
			$  11$   & $-0.082782$  &$ 11$  & $-0.135886$  &$  11$   & $-0.155319$   &$ 11$ & $-0.170848$   \\
			$  12$   & $-0.082808$  &$ 12$  & $-0.136220$  &$  12$   & $-0.155939$   &$ 12$ & $-0.171805$   \\
			\hline                                                                                    
		\end{tabular}
		\caption{Coupled Cluster Doubles correlation energy for QD for two electrons and different values of $\omega$ in atomic units. }   \label{tab:resultsCCD}
	\end{center}	
\end{table}
\begin{table}[h!]
	\begin{center}
		\begin{tabular}{|c c| c c| c c| c c|}
			%\cline{1-12}
			%         & \multicolumn{1}{c}{N=2} & \multicolumn{5}{c}{N=6}  \\
			%\cline{1-12}
			\hline
			\multirow{2}{*}{} & 
			\multicolumn{1}{c}{N=2} \vline& 
			\multicolumn{2}{c}{N=6} \vline&
			\multicolumn{2}{c}{N=12} \vline&
			\multicolumn{2}{c}{N=20} \vline\\
			\hline
			$R$  & $E_{HF}$ & $R$ & $E_{HF}$ & $R$  & $E_{HF}$ &$R$ &  $E_{HF}$  \\
			\hline
			$  3 $   & $3.16269$  &$ 3 $  & $21.593198$  &$  3 $   & $73.765549$   &$ 3$  & $-          $   \\
			$  4 $   & $3.16269$  &$ 4 $  & $20.766919$  &$  4 $   & $70.673849$   &$ 4$  & $177.963297 $   \\
			$  5 $   & $3.16192$  &$ 5 $  & $20.748402$  &$  5 $   & $67.569930$   &$ 5$  & $168.792442 $   \\
			$  6 $   & $3.16192$  &$ 6 $  & $20.720257$  &$  6 $   & $67.296869$   &$ 6$  & $161.339721 $   \\
			$  7 $   & $3.16191$  &$ 7 $  & $20.720132$  &$  7 $   & $66.934745$   &$ 7$  & $159.958722 $   \\
			$  8 $   & $3.16191$  &$ 8 $  & $20.719248$  &$  8 $   & $66.923094$   &$ 8$  & $158.400172 $   \\
			$  9 $   & $3.161909$ &$ 9 $  & $20.719248$  &$  9 $   & $66.912244$   &$ 9$  & $158.226013 $   \\
			$  10$   & $3.161909$ &$ 10$  & $20.719217$  &$  10$   & $66.912035$   &$ 10$ & $158.226030 $   \\
			$  11$   & $3.161909$ &$ 11$  & $20.719216$  &$  11$   & $66.911365$   &$ 11$ & $158.010277 $   \\
			$  12$   & $3.161909$ &$ 12$  & $20.719216$  &$  12$   & $66.911364$   &$ 12$ & $158.004953 $   \\
			\hline                                                                                    
		\end{tabular}
		\caption{ Ground-state energies for QD measured in a.u. obtained numerically by Hartree-Fock method for different number of shells (R) and electrons (N).  Oscillator strength $\omega=1$.}   \label{tab:resultsHF}
	\end{center}
\end{table}


\section{The CCQMC results}
CCQMC is a population dynamics algorithm. That require for it to reproduce a certain behavior for the population of walkers, or in our case excips. That means that the validity of our CCQMC results is based on getting the proper population dynamics for the excips. On the other hand for this particular algorithm the sing structure of excips should be correct. \\
For the HEG we do not have single excitations and have to deal only with doubles. When we perform the computation for the deterministic CCD this was a huge advantage. However, it might not be so for the CCQMC, because it impose limits on our ability to spawn excips in the system. \\
Let's consider how this is done for our simulation. We only have three different cluster sizes in our simulation: cluster size one, two and three. The cluster size one is the cluster with zero excitors and there is only one such cluster, which is reference determinant. As soon as we can spawn only double or single excitation from the selected cluster this leave us with only doubles being spawned from reference. In the beginning of simulation, before we start population control, the death also prohibited for the reference, so it's population never decrease at this point. For the cluster size one we are not able to spawn any excips, because we do not consider quadruples and do not have singles at the same time, so the only possibility here is death for the excips spawned before. After we start population control the probability for excips to die changes, but it still no other outcomes possible on this step. For the cluster size three both spawning and death could be possible is we include quadruples. However we do not have them from previous steps, so death on this step never occurs and the only possible outcome here is to spawn a doubly excited excip. This happens due to the only existing excitors are doubles and that means we can only create a composite cluster combining two doubles and getting a quadrupole excitation. The absence of singles leave us with much less possibilities to spawn and/or kill excips and this introduce some difficulties in analyzing the population dynamics of the simulation.\\
In order to obtain a valid results from the simulation one first have to reach a so-called "plateau". As it have been presented in the article by Thom and Spencer \cite{spencerDevelopmentsStochasticCoupled2016} one can adjust simulation parameters in a way that plateau become easy to spot on a plot. Fig. $\ref{fig:thomEG}$ present the results they have obtained for the neon atom for CCQDSTQ approximation. Despite the fact that we do not take into account higher excitation it could be a valid test if we could reproduce the same behavior in our simulation. The obvious difficulty here is the fact that we have to be vary careful choosing the parameters for the simulation and the fact that we are very limited in our ability to tune them all at the same time. One of the solutions here might be to fix some of them for a while and investigate the behavior of the system. We decide to fix population on reference determinant and run the simulation. The population dynamics should be close to those obtained by Thom and Spencer. Fig. $\ref{fig:nx2k}$ present the population dynamics we got for 14 electrons and 54 states. As one can see from the plot the plateau is reached for the critical number of excips being between $10^3$ and $10^4$, which is a result we expected to get for the $r_s=0.5$. On Fig. $\ref{fig:CCDvsCCDT}$ the difference between CCSD and CCSDT for electron gas decreases for smaller $r_s$. We also run the simulation for 162 basis functions and observe a growth in number of excips - it become closer to $10^4$. From such population dynamics one may assume that we got a proper sign structure. However this results are obtained for fixed normalization and it might introduce some bias in the simulation. \\
On Fig. $\ref{fig:platFind}$ and Fig. $\ref{fig:platFindStune}$ one can see the distribution for Death and Spawn processes for different excips signs. We have already tested the behavior of excips population, but how can we test the sign structure? As soon as our excips represent the amplitudes of the deterministic CCD one can simply compare the signs of excips with the sign of the corresponding amplitude. However the main idea of the method is to use it for a large basis sets and number of particles, when the deterministic method can't be used for large basis sets and large systems. Instead we can claim that for the stable simulation the number of excips spawned or killed with different sings should remain approximately the same and fluctuate around some value. Which is exactly what we see from the Fig. $\ref{fig:platFind}$ and Fig. $\ref{fig:platFindStune}$. The only thing we want to mention here is that we have also checked the signs of excips against the signed of deterministic amplitudes and they agree, except for some poorly populated excitors.\\


\begin{figure}[ht!]
	\centering
	\includegraphics[width=0.8\linewidth]{thomEG}
	\caption{Ne cc-pVQZ CCSDTQ calculations starting with different initial
		particle numbers at the reference and different timesteps. (a): With a carefully
		chosen low timestep and initial population, a plateau is visible. An increased
		timestep and initial population overshoot the plateau but have a shoulder.
		The lower panel shows a maximum of the particle ratio at the position of
		the shoulder and plateau. (b): "Shoulder plots" allow shoulder height to be
		read off easily and calculations compared. Reproduced from \cite{spencerDevelopmentsStochasticCoupled2016}, with the permission of AIP Publishing. }
	\label{fig:thomEG}
\end{figure}


\begin{figure}[ht!]
	\centering
	\includegraphics[width=0.8\linewidth]{Nex2000new}
	\caption{The population dynamics of excips for $N_e=14$, 54 basis functions. $r_s=0.5$, $\delta \tau=0.0005$.}
	\label{fig:nx2k}
\end{figure}




\begin{figure}[ht!]
	\centering
	\includegraphics[width=0.8\linewidth]{Nex20000}
	\caption{The population dynamics of excips for $N_e=14$, 54 basis functions, $r_s=0.5$, $\delta \tau=0.0005$ with the population control. enabled after $5\cdot 10^3$ iterations. Damping parameter  $\gamma = 0.05$ and $A=5$.}
	\label{fig:nx20k}
\end{figure}


\begin{landscape}
	\begin{figure}[ht!]
		\centering
		\includegraphics[width=0.8\linewidth]{platFind}
		\caption{Population dynamics, $N_e=14$, $r_s=0.5$, no population control.}
		\label{fig:platFind}
	\end{figure}
\end{landscape}

\begin{landscape}
	\begin{figure}[ht!]
		\centering
		\includegraphics[width=0.8\linewidth]{platFindStune}
		\caption{Population dynamics, $N_e=14$, $r_s=0.5$, $\delta \tau=0.0001$, 54 basis functions  with population control. Damping parameter $\gamma = 0.05$. }
		\label{fig:platFindStune}
	\end{figure}
	
\end{landscape}


